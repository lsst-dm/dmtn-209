
\begin{abstract}

In 2020 Rubin Observatory made a tender for a three year interim data facility. This was won by Google. In partnership with Google we have now deployed the first externally used version of the Rubin Science Platform consisting of a Jupyter environment and a portal. Sitting behind this is TAP access to our project developed large scale database Qserv holding a synthetic catalogue of 140 million objects  and around 500TB of simulated  images. This is  hosted on Google Cloud Platform  using Terraform and Kubernetes. The initial few hundred users were given access to this in July 2021. In this talk we will describe the system, the initial use and the coming years.
\end{abstract}


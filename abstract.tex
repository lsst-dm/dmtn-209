
\begin{abstract}
In 2020 Rubin Observatory made a tender for a three year interim data facility. This was won by Google. In partnership with Google we have now deployed the first externally used version of the Rubin Science Platform consisting of a jupyter environment and a portal. Sitting behind this is TAP access to our project developed large scale database Qserv holding a synthetic catalogue of 140 million objects  and around 500TB of simulated  images. This is  hosted on Google Cloud Platform  using Terraform and Kubernetes. The initial few hundred users were given access to this in July 2021. In this talk we will describe the system, the initial use and the coming years. 
As the Commissioning Execution Plan (LSE-390) says, "The project team shall
deliver all reports documenting the as-built hardware and software including:
drawings, source code, modifications, compliance exceptions, and recommendations
for improvement." As a first step towards the delivery of documents that will describe the system at the
end of construction, we are assembling teams for producing of the order 40 papers
that eventually will be submitted to relevant professional journals. The immediate goal is to accomplish
all the writing that can be done without data analysis before the data
taking begins, and the team becomes much more busy and stressed.

This document provides the template for these papers.
\end{abstract}


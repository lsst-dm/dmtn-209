\documentclass[11pt,twoside]{article}

% Do NOT use ANY packages other than asp2014.
\usepackage{asp2014}

\aspSuppressVolSlug
\resetcounters

% References must all use BibTeX entries in a .bibfile.
% References must be cited in the text using \citet{} or \citep{}.
% Do not use \cite{}.
% See ManuscriptInstructions.pdf for more details
\bibliographystyle{asp2014}
\def\procspie{Proc.\ SPIE} % Proceedings of the SPIE


% The ``markboth'' line sets up the running heads for the paper.
% 1 author: "Surname"
% 2 authors: "Surname1 and Surname2"
% 3 authors: "Surname1, Surname2, and Surname3"
% >3 authors: "Surname1 et al."
% Replace ``Short Title'' with the actual paper title, shortened if necessary.
% Use mixed case type for the shortened title
% Ensure shortened title does not cause an overfull hbox LaTeX error
% See ASPmanual2010.pdf 2.1.4  and ManuscriptInstructions.pdf for more details
\markboth{Economou, O'Mullane}{Rubin Science Platform on Google Title}

\begin{document}

\title{Rubin Science Platform on Google: the story so far.}

% Note the position of the comma between the author name and the
% affiliation number.
% Authors surnames should come after first names or initials, eg John Smith, or J. Smith.
% Author names should be separated by commas.
% The final author should be preceded by "and".
% Affiliations should not be repeated across multiple \affil commands. If several
% authors share an affiliation this should be in a single \affil which can then
% be referenced for several author names. If only one affiliation, no footnotes are needed.
% See ManuscriptInstructions.pdf and ASP's manual2010.pdf 3.1.4 for more details
\input{authors}

% There should be one \aindex line (commented out) for each author. These are used to
% build up the author index for the Proceedings. The surname must come first, followed by
% initials. Note the use of ~ before each initial to control spacing.
% The \author entries (see above) have surname last. These \aindex entries have
% surname first.
% The Aindex.py command willl create them for you after you have constructed the \author
% The first entry should be the first author, for bold-facing the author index page-reference

%\aindex{FistAuthor1,~S.~A.}
%\aindex{Author2,~S.~B.}
%\aindex{Author3,~S.}



\begin{abstract}

In 2020 Rubin Observatory made a tender for a three year interim data facility. This was won by Google. In partnership with Google we have now deployed the first externally used version of the Rubin Science Platform consisting of a Jupyter environment and a portal. Sitting behind this is TAP access to our project developed large scale database Qserv holding a synthetic catalogue of 140 million objects  and around 500TB of simulated  images. The initial users were given access to this in July 2021.
\end{abstract}





%Slides on https://docs.google.com/presentation/d/15XiQawtQ4aZCwU1qQ6seXXg6jTfLU7j7K9pjh7ReucE/edit#slide=id.ged99d87c95_0_165
\section{Introduction}

The Legacy Survey of Space and Time \citep{2019ApJ...873..111I} is an optical/near-IR survey of half the sky in ugrizy bands to r 27.5 (36 nJy) based on 825 visits over a 10-year period: deep wide fast.
Carried out by Rubin observatory on Cerro Pach\'{o}n Chile, the survey will produce around 100\,PB of data consisting of about a billion 16\,Mpix images, enabling measurements for 40 billion objects!

We have recently begun a preoperations program which includes exposing users to the Science Platform.


\section{Interim Data Facility on Google }
We decided on an Interim Data Facility (IDF) to mitigate the risks of — and the delay imposed by — the US Data Facility(USDF)  selection process.
We felt this an excellent transition activity to prepare the operations  team and Community for Rubin data.
In a competitive tender we selected Google for this three year activity.
We have installed construction delivered services (Rubin Science Platform and middleware/plumbing) on Google Cloud Platform.
The architecture is depicted in Figure \ref{fig:IDFarch}.
We primarily leverage Google Kubernetes Engine (GKE) and Google Cloud Storage (GCS) to scale up and down as needed while batch processing images  and serving the data release products.
The full stack was automated using Terraform with help from the Burwood Group, which is not highlighted in the diagram but is outlined in section \ref{sec:infra}.
In addition running services at scale improves readiness for Data Production activity during full survey operations.




\articlefigure[width=0.9\textwidth]{IDFarch.eps}{fig:IDFarch}{Architecture of the Interim Data Facility on Google, gray Items (alerts) are not yet implemented but shall be in 2022.}



\subsection{ Docker, Kubernetes, Terraform} \label{sec:infra}
The setup of the infrastructure such as clusters on GCP is done using Terraform\footnote{\url{https://www.terraform.io/}}.
Typically we have development, integration and production clusters for each major deployment e.g.\ Science Platform and Qserv.
The orchestration of the Rubin application containers is done using Kubernetes\footnote{\url{https://kubernetes.io/}} with deployment handled by ArgoCD\footnote{\url{https://argoproj.github.io/cd/}}.
Our build system, based on Jenkins and GitHub Actions,  produces Docker containers for deployment \citep[see e.g.,][]{2018SPIE10707E..09J}.

\section {Data Previews}
The IDF will be used for a series of Data Previews \citep[DP;][]{RDO-011}. DP0 is using simulated data, DP1 commissioning camera data,
and DP2 LSSTCam commissioning data. DP0 currently has two phases\citep{RTN-001} with one focussing on data access and the other on data processing.
In June 2021 \href{https://data.lsst.cloud}{data.lsst.cloud} was opened up to the first 200 delegates as part of Data Preview 0.1 with data from the DESC DC2 simulated data \citep{arXiv:2010.05926}.
For Data Preview 0.2 we shall reprocess the DESC data (DP0.1) with the LSST Science Pipelines v23, to generate a fully self-consistent data release.
This will entail producing all the data products and catalogs as well as loading the  catalogs in Qserv.
Thus demonstrating a portable set of cloud enabled tools based on Butler Gen3 \citep{2019ASPC..523..653J} and PanDA.

\section{Community engagement }
The Rubin Community Engagement Team (CET) provided documentation and support to the scientists and students participating in DP0 as delegates.
The term ``DP0 delegates'' was adopted to reflect DP0 participants’ important roles of representing the broad science community as learners, testers, and providers of feedback, and of sharing the benefits of their DP0 participation with their communities as teachers and colleagues.

Documentation at \href{https://dp0-1.lsst.io}{dp0-1.lsst.io} included a delegate's homepage listing all the resources, events, guidelines, and expectations; descriptions of the data products and table schema; instructions for accessing the RSP's three aspects and usage risks; and tutorials to demonstrate simple science investigations with the DP0 data set at the RSP.
Delegate resources included biweekly virtual ``Delegate Assemblies'' with hands-on demonstrations and breakout rooms to facilitate co-working; a dedicated category in the Rubin Community Forum (\url{https://community.lsst.org}) where delegates can ask questions and discuss their DP0 work; and a GitHub repository where delegates can contribute to shared code and notebooks.
GitHub Issues were also used as a means for delegates to request technical assistance.

Rubin staff from the CET and Data Production worked together to respond to delegate's requests for support and bug reports.
All of the delegate resources are publicly accessible to non-delegates, including the assemblies (and their recordings) and the Community Forum discussions.
Some aspects of the Community Engagement model as applied to DP0 are designed to scale to thousands of users for the future Data Previews, such as the extensive documentation, the tutorials, and the use of the Community Forum.
However other aspects, such as the frequent live virtual sessions, would not scale well to thousands of users and were instead specifically designed to ramp up skill levels from novice to intermediate in order to seed expertise with Rubin software in the community.
Looking forward to future Data Previews, the Community Engagement strategy will focus on building and fostering a vibrant science community,
supported by infrastructure, that is able to help itself and to crowd-source solutions.

\section{Conclusion and lessons learned}
User feedback has been very positive on this initial Science Platform.
Users do need a lot of guidance if we are to be efficient.
For example Qserv, accepting ADQL, can build great histograms (minutes for full dataset) but users
tended to do this inPpython by running a query to extract all the data and then binning it in Python.
AutoScale really works -- but it's not really fast considering our containers, it takes minutes to spin up new node -- eons as far as users concerned!
DASK looks very interesting but we are not sure how to manage resource allocations yet.
No matter how much you prepare you always learn something new!
The Data Butler abstracts data access from the user and we designed the Butler to support pluggable backends from the beginning.
Deploying the Butler to Google was relatively straightforward since Butler already supported Postgres for the registry and S3 for file access.

Management buy-in is a big part of any cloud adventure, there is a lot of reluctance to change from in-house models
which requires much convincing. One thing we did were proof of concept studies with both Amazon \citep{2020arXiv201106044B,DMTN-137} and Google \citep{DMTN-125}. Each ran for six months and demonstrated we could deploy and operate on either platform. This allayed concerns about vendor lockin and improved cost models.

Since we built cloud ready services, setup on GCP was fairly painless and smooth.
Support from Google and Burwood was excellent and rapid, and weekly tag-ups continue to make sure everything is in place or planned.
Since Rubin is between data facilities for some years this is a quite cost effective approach.

\bibliography{O3-2}  % For BibTex


\end{document}
